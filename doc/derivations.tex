% homework9.tex

\documentclass[11pt,letterpaper]{article}

%---------------------------------- Packages -----------------------------------
\usepackage{amsmath}
\usepackage{unicode-math}
\usepackage{fontspec}
\usepackage{graphicx}
\usepackage{tabulary}
\usepackage{hyperref}
\usepackage{fancyvrb}

%----------------------------------- Options -----------------------------------
\setmathfont{STIX Two Math}
\setmainfont{STIX Two Text}
\setmonofont{Consolas}

\hypersetup{
  colorlinks  = true,
  urlcolor    = blue,
  linkcolor   = blue,
  citecolor   = blue
}

%-------------------------------- Document info --------------------------------
\title{Story Spring Derivation \\
       \vspace{0.5em}
       \large Modified Ibarra-Medina-Krawinkler Bilinear material\\
       \vspace{0.5em}
       \large ASCE 7-10 Equivalent Lateral Force Procedure}
\author{Peter Talley}
\date{April 25, 2017}

%###############################################################################
%################################ Document Body ################################
%###############################################################################
\begin{document}

\maketitle

%---------------------------------- Section 1 ----------------------------------
\section{Design story stiffness}
The deflection of a given story $x$ is given by equation 12.8-15 as:

\begin{equation} \label{eq:defl}
    \delta_x = \frac{C_d \delta_{xe}}{I_e}
\end{equation}
Where $\delta_{xe}$ is the calculated elastic deflection. This is taken to be:
\begin{equation} \label{eq:elastdefl}
	\delta_{xe} = \frac{V_x}{K_x} + \delta_{(x-1)e}
\end{equation}
Where $V_x$ is the design story shear (\S 12.8.3), $K_x$ is the story stiffness, and $\delta_{(x-1)e}$ is the elastic deflection of the story below $x$. The story drift is the difference between story deflections:
\begin{equation} \label{eq:drift}
	\Delta_x = \delta_x - \delta_{x-1}
\end{equation}
Taking $\Delta_x$ to be the allowable story drift, $\Delta_a$, and substituting equations \eqref{eq:defl} and \eqref{eq:elastdefl} into \eqref{eq:drift} obtains:
\begin{equation} \label{eq:driftsubs}
	\Delta_a = \frac{C_d}{I_e} \left(\frac{V_x}{K_x}+\delta_{(x-1)e}-\delta_{(x-1)e}\right)
\end{equation}
$K_x$ is then solved for:
\begin{equation} \label{eq:stiffness}
    K_x = \frac{C_d V_x}{I_e \Delta_a}
\end{equation}

%---------------------------------- Section 2 ----------------------------------
\section{Design story strength}
The maximum strength of each story $x$, $V_c$, is taken as:
\begin{equation}
  V_{c} = Ω_o V_x
\end{equation}
Where $Ω_o$ is the system overstrength factor and $V_x$ is the design story shear.

%---------------------------------- Section 3 ----------------------------------
\section{Pre-capping deflection}
The design strength, $V_c$, is connected to the yield strength, $V_y$, by the parameter $C_{yc}$:
\begin{equation} \label{eq:C_yc}
  V_y = C_{yc}V_c
\end{equation}
The deflection at yield and pre-capping deflection can now be calculated. Using linear relations and equation \eqref{eq:C_yc}, the following equations are developed:
\begin{align}
  Δ_{y} &= \frac{V_y}{K_x}         \label{eq:deflecty} \\
  Δ_{p} &= \frac{V_c-V_y}{a_sK_x}  \label{eq:deflectp}
\end{align}
These equations are evaluated for each floor $x$. $a_s$ and $C_{yc}$ are assumed to be the same for each story.

%---------------------------------- Section 4 ----------------------------------
\section{Post-capping and ultimate deflection}
The post-capping deflection is currently defined as proportional to $Δ_p$:
\begin{equation}
  Δ_{pc} = C_{pcp}Δ_p
\end{equation}
The ultimate deflection is currently defined as proportional to the sum of the previously calculated deflections:
\begin{equation}
  Δ_u = C_{upc}\left(Δ_y + Δ_p + Δ_{pc}\right)
\end{equation}

\end{document}
