% homework9.tex

\documentclass[11pt,letterpaper]{article}

%---------------------------------- Packages -----------------------------------
\usepackage{amsmath}
\usepackage{unicode-math}
\usepackage{fontspec}
\usepackage{graphicx}
\usepackage{tabulary}
\usepackage{hyperref}
\usepackage{fancyvrb}

%----------------------------------- Options -----------------------------------
\setmathfont{STIX Two Math}
\setmainfont{STIX Two Text}
\setmonofont{Consolas}

\hypersetup{
  colorlinks  = true,
  urlcolor    = blue,
  linkcolor   = blue,
  citecolor   = blue
}

%-------------------------------- Document info --------------------------------
\title{Story Spring Derivation \\
       \vspace{0.5em}
       \large Modified Ibarra-Medina-Krawinkler Bilinear material\\
       \vspace{0.5em}
       \large ASCE 7-10 Equivalent Lateral Force Procedure}
\author{Peter Talley}
\date{April 18, 2017}

%###############################################################################
%################################ Document Body ################################
%###############################################################################
\begin{document}

\maketitle

%---------------------------------- Section 1 ----------------------------------
\section{Design story stiffness}
The deflection of a given story $x$ is given by equation 12.8-15 as:

\begin{equation} \label{eq:defl}
    \delta_x = \frac{C_d \delta_{xe}}{I_e}
\end{equation}
Where $\delta_{xe}$ is the calculated elastic deflection. This is taken to be:
\begin{equation} \label{eq:elastdefl}
	\delta_{xe} = \frac{F_x}{K_x}
\end{equation}
Where $F_x$ is the design story shear (\S 12.8.3) and $K_x$ is the story stiffness. The story drift is the difference between story deflections:
\begin{equation} \label{eq:drift}
	\Delta_x = \delta_x - \delta_{x-1}
\end{equation}
Equations \eqref{eq:defl} and \eqref{eq:elastdefl} can be substituted into this to get:
\begin{equation} \label{eq:driftsubs}
	\Delta_x = \frac{C_d}{I_e} \left(\frac{F_x}{K_x} - \frac{F_{x-1}}{K_{x-1}}\right)
\end{equation}
$K_x$ is then solved for:
\begin{equation} \label{eq:stiffness}
    K_x = \frac{F_x}{\frac{\Delta_x I_e}{C_d} + \frac{F_{x-1}}{K_{x-1}}}
\end{equation}

\section{Design story strength}
The maximum strength of each story $x$, $M_c$, is taken as:
\begin{equation}
  M_{cx} = Ω_o F_x
\end{equation}
Where $Ω_o$ is the system overstrength factor and $F_x$ is the design story shear.

\section{Pre-capping rotation}
The equivalent lateral force procedure returns the maximum strength, $M_c$, which is used to back-calculate the yield strength, $M_y$, of the story. This also requires an assumption of the ratio of the pre-capping rotation, $θ_p$, to the rotation at yield, $θ_y$. This is treated as a variable in the analysis, $C_{py}$ or \texttt{py\_factor}. Using linear relations and this proportional assumption, the following system of equations is developed:
\begin{subequations}
  \label{eq:rotation}
  \begin{align}
    θ_{px} - C_{py}θ_{yx} & = 0 \label{eq:rotate1} \\
    a_sK_xθ_{px} + M_{yx} & = M_{cx} \label{eq:rotate2} \\
    θ_{yx} - K_x^{-1}M_{yx} & = 0 \label{eq:rotate3}
  \end{align}
\end{subequations}
This system is solved for each floor $x$. $a_s$ and $C_{py}$ are assumed to be the same for each story.

\section{Post-capping and ultimate rotation}
The post-capping rotation is currently defined as proportional to $θ_p$:
\begin{equation}
  θ_{pc} = C_{pcp}θ_p
\end{equation}
The ultimate rotation is currently defined simply in terms of the previously calculated rotations, taking some amount of $θ_{pc}$:
\begin{equation}
  θ_u = θ_y + θ_p + C_{pcu}θ_{pc}
\end{equation}

\end{document}
